% !TEX root = main.tex

\documentclass[letterpaper, 11pt]{extarticle}
% \usepackage{fontspec}

% ==================================================

% document parameters
% \usepackage[spanish, mexico, es-lcroman]{babel}
\usepackage[english]{babel}
\usepackage[margin = 1in]{geometry}

% ==================================================

% Packages for math
\usepackage{mathrsfs}
\usepackage{amsfonts}
\usepackage{amsmath}
\usepackage{amsthm}
\usepackage{amssymb}
\usepackage{physics}
\usepackage{dsfont}
\usepackage{esint}

% ==================================================

% Packages for writing
\usepackage{enumerate}
\usepackage[shortlabels]{enumitem}
\usepackage{framed}
\usepackage{csquotes}

% ==================================================

% Miscellaneous packages
\usepackage{float}
\usepackage{tabularx}
\usepackage{xcolor}
\usepackage{multicol}
\usepackage{subcaption}
\usepackage{caption}
\captionsetup{format = hang, margin = 10pt, font = small, labelfont = bf}

% Citation
\usepackage[round, authoryear]{natbib}

% Hyperlinks setup
\usepackage{hyperref}
\definecolor{links}{rgb}{0.36,0.54,0.66}
\hypersetup{
   colorlinks = true,
    linkcolor = black,
     urlcolor = blue,
    citecolor = blue,
    filecolor = blue,
    pdfauthor = {Author},
     pdftitle = {Title},
   pdfsubject = {subject},
  pdfkeywords = {one, two},
  pdfproducer = {LaTeX},
   pdfcreator = {pdfLaTeX},
   }

% \begin{document}
% Hello, world! This is a minimal document to test the preamble.
% \end{document}

\usepackage{listings} 
\usepackage{xcolor}

\definecolor{codegreen}{rgb}{0.12,0.6,0.33}
\definecolor{codegray}{rgb}{0.5,0.5,0.5}
\definecolor{codepurple}{rgb}{0.58,0,0.82}
\definecolor{codeblue}{rgb}{0,0.1,0.8}
\definecolor{backcolour}{rgb}{0.95,0.95,0.92}

\lstdefinestyle{pythonstyle}{
   backgroundcolor=\color{backcolour},
   commentstyle=\color{codegreen},
   keywordstyle=\color{codeblue}\bfseries,
   numberstyle=\tiny\color{codegray},
   stringstyle=\color{codepurple},
   basicstyle=\ttfamily\footnotesize,
   breakatwhitespace=false,
   breaklines=true,
   captionpos=b,
   keepspaces=true,
   numbers=left,
   numbersep=5pt,
   showspaces=false,
   showstringspaces=false,
   showtabs=false,
   tabsize=4,
   language=Python
}

% \usepackage{minted}

\input{format}
\input{commands}

\begin{document}

\textsf{\LARGE{\textbf{Take-home exam 1}}}

\normalsize{\textit{Dynamical Systems}}

\vspace{1ex}

\textsf{\textbf{Student:}} \text{Males-Araujo Yorlan}, 
\href{mailto:yorlan.males@yachaytech.edu.ec}{\texttt{yorlan.males@yachaytech.edu.ec}}\\
\textsf{\textbf{Lecturer:}} \text{Mario Cosenza}, 
\href{mcosenza@yachaytech.edu.ec}{\texttt{mcosenza@yachaytech.edu.ec}}

\vspace{2ex}

\begin{problem}{Classification of fixed points}{problem-label}
A particle of mass $m = 1$ is moving in the potential $V(x) = - (1/2)x^2 + (1/4)x^4$.
Find and classify the fixed points (node, saddle, focus) according to their stability.
\end{problem}

Solution goes here.

\begin{problem}{Hopf bifurcation}{problem-label-2}
Consider the system $\ddot{x} + \lambda(x^2 - 1)\dot{x} + x - a = 0$.
Find the curves on the space of parameters $(\lambda, a)$ where a Hopf bifurcation occurs.
\end{problem}

The second solution goes here.

\begin{problem}{Fractal dimension}{problem-label-3}
Calculate the fractal dimension of the following object
shown at three successive levels of construction.

\begin{center}
    \includegraphics[scale=0.5]{images/cube_fractal.jpg}
\end{center}
\end{problem}

Solution goes here.

\begin{problem}{Sensitivity and analytical solution}{problem-label-4}

    Consider the map $x_{n+1} = f(x_n) = (2x_n - 1)^3$ , for $x_n \in [-1, 1]$.

    \begin{enumerate}[(a)]
        \item Show, by iterating two close initial conditions, that this map is chaotic.
        \item Show that $x_n = \cos^3 (2^n \cos^{-1} (x_0))$ is a solution $\forall n$.
    \end{enumerate}

\end{problem}

Solution goes here.

\begin{problem}{Bifurcation diagram and Lyapunov exponent}{problem-label-5}
    Consider the map $x_{n+1} = f(x_n) = \sin^2(r\,\arcsin{\sqrt{x_n}})$, for $x_n \in [0, 1]$.

    \begin{enumerate}[(a)]
        \item Obtain the bifurcation diagram of $x_n$ as a function of $r$,
        for $r \in [1, 4]$.
        \item Calculate the Lyapunov exponent as a function of $r$, for $r \in [1, 4]$.
    \end{enumerate}

\end{problem}

Solution goes here.

\begin{problem}{Phase space}{problem-label-6}

    The evolution of a system is described by the following equation:
    \[
        \ddot{x} + a\ddot{x}+\dot{x}-|x|+1=0, \text{ for } a > 0.
    \]

    \begin{enumerate}[(a)]
        \item Find the fixed points of this system.
        \item Plot the attractor of this system in its phase space for $a = 0.6$.
        Is it strange?
        \item Show that this system is not chaotic for $a = 0.68$.
    \end{enumerate}

\end{problem}

Solution goes here.

% Now, let's see how to use the \texttt{problem} environment. 
% The \texttt{problem} environment is defined in the \texttt{format.tex} file. 
% You can define your own environments following 
% the \texttt{problem} environment.

% What do you think about this? I think it's a great 
% way to work with problems. You can also use 
% the \texttt{problem} environment to write your own problems. 
% For instance, you can write your own problems 
% in the \texttt{problem} environment and then use 
% the \texttt{problem} environment to write your own solutions.


% Take this cool equation for example:
% \begin{equation}
%     \label{eq:example}
%     \begin{aligned}
%         \im \hbar \pdv{\psi}{t} &= -\dfrac{\hbar^2}{2m} \nabla^2 \psi + V(x) \psi,
%     \end{aligned}
% \end{equation}

% where $\psi$ is the wave function, $V(x)$ is the potential energy, 
% and $m$ is the mass of the particle. 
% The equation describes how the wave function evolves over time.
% The \texttt{problem} environment is defined in the \texttt{format.tex} file. 
% You can define your own environments following 
% the \texttt{problem} environment.

\vspace{5ex}
\hrule
\vspace{1ex}

\vspace{0.1ex}

% =================================================

% \newpage

% \vfill

% \bibliographystyle{apalike}
% \bibliography{references}

\end{document}